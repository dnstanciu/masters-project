\abstract{
	One of the most common causes of dementia is \ac{AD}. \ac{MEG} is a neuroimaging technique that can monitor how neural oscillations are altered in patients suffering from this disorder. Graph theory is a framework that only recently has been applied to the study of structural and \ac{FC} in the human brain. This project aims to construct and compare functional connectivity graphs of healthy subjects and of patients suffering from \ac{AD} and \ac{MCI}. The debiased weighted phase lag index is used to quantify \ac{PS} between \ac{MEG} sensors. These partial results are used to build \ac{FC} graphs for each subject. A number of graph features such as clustering coefficients, modularity and \ac{SW} topology are compared between the three groups. Classifiers based on logistic regression and random forests are then trained using the computed graph metrics. Recordings of new subjects are assigned into one of the \ac{AD}, \ac{MCI} or \ac{CS} categories. A data processing pipeline that achieves the above was designed in the course of this project. A decrease of the \ac{SW} measure in \ac{AD} networks compared to \ac{CS} indicates a decline in network organisation. Poor sensitivity and specificity render the classifiers unsuitable in clinical settings.
}

% aims and objective
% background 
% methodology - main methods
% results - main findings
% conclusions - main conclusions 

