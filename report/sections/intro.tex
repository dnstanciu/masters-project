\chapter{Introduction}

% 500-750 words

% context in which research took place
% why is the subject important
% key participants




\section{Motivation}
% why was this research carried out; application of machine learning to clinical setting...
\ac{AD} is the most frequent cause of dementia \autocite{Cummings2004}. Global costs are expected to have a remarkable growth as predictions show that by the year 2050, 1 out of 85 people will suffer from the disease \autocite{Brookmeyer2007}. \ac{MCI} is a key pre-stage of \ac{AD} \autocite{Morris2001}. Early diagnosis of this stage using inexpensive biomarkers would allow a larger number of patients to start treatments in advance. This may significantly delay disease onset \autocite{Cummings2004}.

Network analysis of neurodegenerative disorders is a young research field which has already shown promising results in disease diagnosis and progression \autocite{Zhou2010, Rowe2010DCM}. \ac{MEG} is a neuroimaging technique that can help find differences between neural patterns in the diseased brain and healthy brain \autocite{Stam2010}. Graph theory is a well-established framework that has the potential to standardise interpretation of neuroimaging results \autocite{Stam2010}. 

Existing results are inconsistent with respect to certain methodologies \autocite{Tijms2013}. This piece of research aims to supplement the current \ac{AD} research literature by employing promising techniques from three areas of interest: signal processing, graph theory and machine learning. 

\section{Objectives}
\label{sec:objectives}

The present data analysis project aimed to answer the following research question: 

\begin{center}
	{\textbf{What are the identifiable differences in functional brain connectivity between populations of \ac{AD}, \ac{MCI} and \ac{CS}?}}
\end{center}
The partial objectives of the project were defined in the original proposal \autocite{Stanciu} as follows:


\begin{description}
\item[Signal Processing] \hfill \\
  Compute correlations between pairs of \ac{MEG} sensors
  \item[Graph Analysis] \hfill \\
  Compute functional connectivity graphs and graph measures
  \item[Classification] \hfill \\
  Train classifiers using graph measures as features to assign recordings of new subjects in one of the \ac{AD}, \ac{MCI} or \ac{CS} categories
\end{description}
% end Objectives

\section{Achievements}

A modular data processing pipeline has been created which is able to take \ac{MEG} recordings as input and make predictions about which of the \ac{AD}, \ac{MCI} or \ac{CS} groups the signals originated from.

Brain connectivity graphs were created and graph measures were extracted from them. Classifiers were trained on these measures with the purpose of disease prediction.

Statistical analysis showed that for certain pipeline settings, results were not significant. Nevertheless, an easy to customise framework is in place and future methodological refinements should produce better results.

% end Achievements

\section{Outline}

% what topics are going to be discussed in each chapter and how each chapter related to each other; ROADMAP 

\begin{itemize}
	\item \textbf{Chapter 2} contains background research for better understanding of the scope of the project. Brief presentations of \ac{AD} and \ac{MEG} are provided in the beginning sections. These are followed by a description of graph theory concepts and a review of recent studies applying graph theory in the context of functional brain connectivity.

	\item \textbf{Chapter 3} describes the signal processing, graph theory and classification stages of the pipeline used to address the research question in Section~\ref{sec:objectives}. Techniques used for each of the three modules are explained. 

	\item \textbf{Chapter 4} presents the results of this project. These are comprised of extracted graph measures, statistical analysis performed on these values and performances of classification methods.

	\item \textbf{Chapter 5} evaluates the findings and discusses project limitations. 

	\item \textbf{Chapter 6} compares deliverables with initial project objectives and suggests avenues for future research.   
\end{itemize}
% end Outline

\section*{Conclusion}

This chapter described why research in \ac{AD} is important and outlined the project objectives. The next chapter describes key concepts needed for a better appreciation of the research methods used in this project. 








